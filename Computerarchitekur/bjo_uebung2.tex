\documentclass[]{article}
\usepackage{lmodern}
\usepackage{amssymb,amsmath}
\usepackage{ifxetex,ifluatex}
\usepackage{fixltx2e} % provides \textsubscript
\ifnum 0\ifxetex 1\fi\ifluatex 1\fi=0 % if pdftex
  \usepackage[T1]{fontenc}
  \usepackage[utf8]{inputenc}
\else % if luatex or xelatex
  \ifxetex
    \usepackage{mathspec}
  \else
    \usepackage{fontspec}
  \fi
  \defaultfontfeatures{Ligatures=TeX,Scale=MatchLowercase}
\fi
% use upquote if available, for straight quotes in verbatim environments
\IfFileExists{upquote.sty}{\usepackage{upquote}}{}
% use microtype if available
\IfFileExists{microtype.sty}{%
\usepackage{microtype}
\UseMicrotypeSet[protrusion]{basicmath} % disable protrusion for tt fonts
}{}
\usepackage{hyperref}
\hypersetup{unicode=true,
            pdfborder={0 0 0},
            breaklinks=true}
\urlstyle{same}  % don't use monospace font for urls
\usepackage{longtable,booktabs}
\IfFileExists{parskip.sty}{%
\usepackage{parskip}
}{% else
\setlength{\parindent}{0pt}
\setlength{\parskip}{6pt plus 2pt minus 1pt}
}
\setlength{\emergencystretch}{3em}  % prevent overfull lines
\providecommand{\tightlist}{%
  \setlength{\itemsep}{0pt}\setlength{\parskip}{0pt}}
\setcounter{secnumdepth}{0}
% Redefines (sub)paragraphs to behave more like sections
\ifx\paragraph\undefined\else
\let\oldparagraph\paragraph
\renewcommand{\paragraph}[1]{\oldparagraph{#1}\mbox{}}
\fi
\ifx\subparagraph\undefined\else
\let\oldsubparagraph\subparagraph
\renewcommand{\subparagraph}[1]{\oldsubparagraph{#1}\mbox{}}
\fi
\DeclareUnicodeCharacter{263A}{:)}
\date{}

\begin{document}

Barnabas Jovanovics

\subsection{1a) 14}\label{a-14}

\begin{enumerate}
\tightlist
\item
  In Binaer umwandeln 14 -> 1110
\item
  Normalisierung 1110 = 1.110 * 2\^{3}
\item
  VZ-Bit: + -> 0
\item
  Exponent: +3 +127 -> 10000010
\item
  Fraction: M = 1.11 F = 11000000000000000000000
\item
  Bitmuster:\\
  1-Vz-Bit 8-Bit-Exponent 23-Bit Fraction\\
  0100|0001|0110|0000|0000|0000|0000|0000
\item
  HEX-Darstellung\\
  4 1 6 0 0 0 0 0
\end{enumerate}

\subsection{1b) 13/32}\label{b-1332}

\begin{enumerate}
\tightlist
\item
  In Binaer umwandeln 13/32 = 13 * 2\^{}-5 -> 0.01101
\item
  Normalisierung 0.01101 = 1.101 * 2\^{}-2
\item
  VZ-Bit: + -> 0
\item
  Exponent: -2 +127 -> 01111101
\item
  Fraction: M = 1.101 F = 10100000000000000000000
\item
  Bitmuster:\\
  1-Vz-Bit 8-Bit-Exponent 23-Bit Fraction\\
  0011|1110|1101|0000|0000|0000|0000|0000
\item
  HEX-Darstellung\\
  3 E D 0 0 0 0 0
\end{enumerate}

\subsection{1c) -13/32}\label{c--1332}

\begin{enumerate}
\tightlist
\item
  In Binaer umwandeln 13/32 = 13 * 2\^{}-5 -> -0.01101
\item
  Normalisierung -0.01101 = -1.101 * 2\^{}-2
\item
  VZ-Bit: - -> 1
\item
  Exponent: -2 +127 -> 01111101
\item
  Fraction: M = 1.101 F = 10100000000000000000000
\item
  Bitmuster:\\
  1-Vz-Bit 8-Bit-Exponent 23-Bit Fraction\\
  1011|1110|1101|0000|0000|0000|0000|0000
\item
  HEX-Darstellung\\
  B E D 0 0 0 0 0
\end{enumerate}

\subsection{1d) 7.125}\label{d-7.125}

\begin{enumerate}
\tightlist
\item
  0.125 * 2 0\\
  0.25 * 2 0\\
  0.5 * 2 1\\
  111.001
\item
  Normalisierung 111.001 = 1.11001 * 2\^{}2
\item
  VZ-Bit: + -> 0
\item
  Exponent: +2 +127 -> 10000001
\item
  Fraction: M = 1.101 F = 11001000000000000000000
\item
  Bitmuster:\\
  1-Vz-Bit 8-Bit-Exponent 23-Bit Fraction\\
  0100|0000|1110|0100|0000|0000|0000|0000
\item
  HEX-Darstellung\\
  4 0 E 4 0 0 0 0
\end{enumerate}

\subsection{1e) -42}\label{e--42}

\begin{enumerate}
\tightlist
\item
  In Binaer umwandeln -101010
\item
  Normalisierung -101010 = -1.0101 * 2\^{}5
\item
  VZ-Bit: - -> 1
\item
  Exponent: +5 +127 -> 10000100
\item
  Fraction: M = 1.101 F = 01010000000000000000000
\item
  Bitmuster:\\
  1-Vz-Bit 8-Bit-Exponent 23-Bit Fraction\\
  1100|0010|0010|1000|0000|0000|0000|0000
\item
  HEX-Darstellung\\
  C 2 2 8 0 0 0 0
\end{enumerate}

\pagebreak

\subsection{2a) 42280000}\label{a-42280000}

\begin{enumerate}
\tightlist
\item
  In Binaer umwandeln
  0|01000100|01010000000000000000000
\item
  Fraction: 01010000000000000000000, Mantisse 1.0101
\item
  Exponent: 01000100 = 132 - 127 = 5
\item
  Vz-Bit 0 -> +
\item
  Normalisierung 1.0101 * 2\^{}5 = 101010
\item
  In Dezimal umwandeln 101010 = 42
\end{enumerate}

\subsection{2b) BFC80000}\label{b-bfc80000}

\begin{enumerate}
\tightlist
\item
  In Binaer umwandeln
  1|01111111|10010000000000000000000
\item
  Fraction: 10010000000000000000000, Mantisse 1.1001
\item
  Exponent: 01111111 = 127 - 127 = 0
\item
  Vz-Bit 1 -> -
\item
  Normalisierung -1.1001
\item
  In Dezimal umwandeln -1.1001 =\\
  1/2+1/16 = 0.5625\\
  -1.5625
\end{enumerate}

\subsection{2c) C18A0000}\label{c-c18a0000}

\begin{enumerate}
\tightlist
\item
  In Binaer umwandeln
  1|10000011|00010100000000000000000
\item
  Fraction: 00010100000000000000000, Mantisse 1.000101
\item
  Exponent: 10000011 = 131 - 127 = 4
\item
  Vz-Bit 1 -> -
\item
  Normalisierung -10001.01
\item
  In Dezimal umwandeln -10001.01 =\\
  1/4 = 0.25\\
  -17.25
\end{enumerate}

\pagebreak

\subsection{3 3EE00000 + 3F880000}\label{ee00000-3f880000}

\begin{enumerate}
\tightlist
\item
  In Binaer umwandeln
  0|01111101|11000000000000000000000
\item
  Fraction: 1100000000000000000000, Mantisse 1.11
\item
  Exponent: 01111101 = 125 - 127 = -2
\item
  Vz-Bit 0 -> +
\item
  Normalisierung 1.11 * 2\^{}-2 -> 0.0111
\end{enumerate}

0.0111

\begin{enumerate}
\tightlist
\item
  In Binaer umwandeln
  0|01111111|00010000000000000000000
\item
  Fraction: 00010000000000000000000, Mantisse 1.0001
\item
  Exponent: 01111111 = 131 - 127 = 0
\item
  Vz-Bit 0 -> +
\item
  Normalisierung 1.0001
\end{enumerate}

1.0001

1.0001\\
0.0111\\
1.1000

0011|1111|1100|0000|0000|0000|0000|0000\\
3FC0000

\pagebreak

\subsection{Ueberschuss-3-Code :)}\label{ueberschuss-3-code}

\begin{longtable}[l]{@{}llll@{}}
\toprule
UTF-8 Byte(s) & Code point & Looks like\\
\midrule
\endhead
c3 9c & U+dc & Ü\\
62 & U+62 & b\\
65 & U+65 & e\\
72 & U+72 & r\\
73 & U+73 & s\\
63 & U+63 & c\\
68 & U+68 & h\\
75 & U+75 & u\\
c3 9f & U+df & ß\\
2d & U+2d & -\\
33 & U+33 & 3\\
2d & U+2d & -\\
43 & U+43 & C\\
6f & U+6f & o\\
64 & U+64 & d\\
65 & U+65 & e\\
20 & U+20 &\\
e2 98 ba & U+263a & ☺\\
\bottomrule
\end{longtable}

36 UNICODE\\
22 UTF-8\\
14



\subsection{5)}\label{Aufgabe 5}

\begin{longtable}[l]{@{}|l|r|r|r|r|@{}}
\toprule
Datentyp & Byte 0 (37) & Byte 1 (E2) & Byte 2 (82) & Byte 3 (AC)\\
\midrule
\endhead
unsigned char & 55 & 226 & 130 &172\\  
signed char & 55 & -30 & -126 &-84\\
\hline
unsigned int & 
\multicolumn{2}{@{}|r|@{}}{14 306} &
\multicolumn{2}{@{}|r|@{}}{33 452}\\
signed int & 
\multicolumn{2}{@{}|r|@{}}{14 306} &
\multicolumn{2}{@{}|r|@{}}{-32 084}\\
\hline
unsigned long &
\multicolumn{4}{@{}|r|@{}}{937  591 468}\\
unsigned long &
\multicolumn{4}{@{}|r|@{}}{937 591 468}\\
float &
\multicolumn{4}{@{}|r|@{}}{2.7002148E-5}\\
\bottomrule
\end{longtable}

\end{document}
